\begin{problem}{gdz}{3s}{256 MiB}{stdin}{stdout}
Đây là một bài toán tương tác.

Byteasar có một hoán vị $(P_0, P_1, \dots,P_{n-1})$ của các số từ $0$ đến $n - 1$.

Byteasar yêu cầu bạn đoán vị trí của số 1 trong hoán vị này. Để bạn không phải đoán một cách mù quáng, 
Byteasar sẽ trả lời câu hỏi ”Ước số chung lớn nhất của hai số $P_i$ và $P_j$ là bao nhiêu?”

Để không quá đơn giản, tối đa bạn có thể hỏi $\left \lfloor \frac{5n}{2} \right \rfloor$ câu hỏi. 
Ngoài ra, Byteasar liên tục đưa ra các hoán vị mới, vì vậy bạn có thể phải trả lời nhiều lần.

\cpsec{Tương tác}

Bạn viết một chương trình giao tiếp với Byteasar bằng cách sử dụng thư viện \texttt{gdzlib.h}, 
đầu tiên hãy dùng chỉ dẫn:

\begin{minted}{cpp}
#include "gdzlib.h"
\end{minted}

Thư viện này cung cấp các hàm sau:

\begin{itemize}
  \item \mint{cpp}{int GetN()} 
  Trả về tham số $n \ (3 \le n \le 5 \times 10^5)$ hoặc $n = -1$. 
  Nếu $n \le 3$ thì đó là độ dài hoán vị hiện tại. Nếu $n = -1$, nghĩa là Byteasar đã hết hoán vị.
  \item \mint{cpp}{int Ask(int i, int j)}
  Hỏi ước số chung lớn nhất của hai số $P_i$ và $P_j$ $(0 \le i, j < n, i \ne j)$.
  Trong một test, hàm này được gọi tối đa $\left \lfloor \frac{5n}{2} \right \rfloor$ lần.
  \item \mint{cpp}{void Answer(int x)}
  Trả lời câu hỏi của Byteasar với $x$ là vị trí của số 1 trong hoán vị hiện tại.
\end{itemize}

Mỗi lần chạy chương trình của bạn bao gồm ít nhất một test. Chương trình phải bắt
đầu bằng \texttt{GetN()} và kết thúc bằng \texttt{GetN()}. Sử dụng \texttt{Ask()} hoặc \texttt{Answer()}.

Sau bộ test cuối cùng, \texttt{GetN()} trả về $-1$. Khi đó, bạn phải chấm dứt chương trình.

Các test có thể có tham số $n$ khác nhau trong một lần chạy chương trình. Tuy nhiên, số
lượng hoán vị trong quá trình thực hiện chương trình sẽ không vượt quá $5 \times 10^5$.

Sử dụng bất kỳ hàm nào có đối số không hợp lệ sẽ bị Wrong Answer.

\end{problem}